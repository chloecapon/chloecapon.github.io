\documentclass[11pt,a4paper,sans]{moderncv}

%% ModernCV themes
\moderncvstyle{classic}
\moderncvcolor{blue}
\renewcommand{\familydefault}{\sfdefault}
\nopagenumbers{}

%% Character encoding
\usepackage[utf8]{inputenc}

%% Adjust the page margins
\usepackage[scale=0.77]{geometry}
\setlength{\hintscolumnwidth}{2.5cm}

\AtEndPreamble{\hypersetup{colorlinks, urlcolor=blue}}

\name{Chloé}{Capon}
\title{PhD student in Mathematics}
\email{chloe.capon@umons.ac.be}
%\photo[70pt][0.4pt]{Photo.jpg}
\extrainfo{\homepagesymbol\httpslink{chloecapon.github.io/}}
\quote{Academic CV – updated on April 26, 2024.}
\begin{document}
\makecvtitle
\section{Research experience}
\cventry{Sept 2022 -- Present}{PhD thesis}{UMONS}{Belgium.}{}{Title: Refinement mechanisms in many-sided reactive synthesis.\\ Supervisor: \href{http://math.umons.ac.be/staff/Randour.Mickael/}{Mickael Randour.}}
\cventry{Sept 2021 -- Jun 2022}{Master’s thesis}{UMONS}{Belgium.}{}{Title: Generation and exploitation of counterexamples in stochastic models.\\ Supervisor: \href{http://math.umons.ac.be/staff/Randour.Mickael/}{Mickael Randour.}}
\cventry{Sept 2021 -- Nov 2021}{Internship}{AISIN Europe}{Belgium.}{}{Subject: Vehicle Routing Problem.}
\cventry{Sept 2020 -- Jun 2021}{Master's project}{UMONS}{Belgium.}{}{Subject: Counterexample generation in stochastic models.\\ Supervisor: \href{http://math.umons.ac.be/staff/Randour.Mickael/}{Mickael Randour.}}
\section{Education}
\cventry{2020 -- 2022}{Master’s degree in Mathematics -- specialist focus: Computer Science}{UMONS}{Belgium.}{}{With highest honours. Received the Maurice Boffa Award from the Department of Mathematics.}
\cventry{2017 -- 2020}{Bachelor's degree in Mathematics}{UMONS}{Belgium.}{}{Academic minor in Computer Science. With distinction.}
\section{Publications}
\closesection{}
Extended versions are available on arXiv (links on \href{http://math.umons.ac.be/staff/Capon.Chloe}{my website}).
\emptysection{}
\subsection{Peer-reviewed proceedings}
[CLW22] Chloé Capon, Nicolas Lecomte, Jef Wijsen. \textbf{Computing H-Partitions in ASP and Datalog.} International Conference on Logic Programming 2022 Workshops co-located with the 38th International Conference on Logic Programming (ICLP 2022), CEUR Workshop Proceedings, volume 3193, 15 pages, 2022.

\section{Attended research events}
\cvitem{2022}{ASPOCP 2022: 15th Workshop on Answer Set Programming and Other Computing Paradigms (Technion campus, Haifa, Israel).}

\section{Teaching}
\cventry{Sept 2022 -- Present}{Complex Analysis}{Teaching assistant}{UMONS}{Belgium.}{Exercise sessions for the course given by \href{https://staff.umons.ac.be/Matthieu.SIMON/}{Matthieu Simon.}}
\cventry{Sept 2022 -- Present}{Elementary Mathematics}{Teaching assistant}{UMONS}{Belgium.}{Supervision of exercise sessions and test grading.}
\cventry{Sept 2022 -- Present}{Introduction to Differential Varieties}{Teaching assistant}{UMONS}{Belgium.}{Exercise sessions for the course given by \href{https://staff.umons.ac.be/Thomas.BRIHAYE/}{Thomas Brihaye.}}
\cventry{Sept 2022 -- Present}{Mathematics - Supplementary Course}{Teaching assistant}{UMONS}{Belgium.}{Exercise sessions for the course given by \href{https://staff.umons.ac.be/Thomas.BRIHAYE/}{Thomas Brihaye.}}
\cventry{Sept 2022 -- Present}{Mathematics for Computer Science}{Teaching assistant}{UMONS}{Belgium.}{Exercise sessions for the course given by \href{https://staff.umons.ac.be/Thomas.BRIHAYE/}{Thomas Brihaye.}}




\section{Languages}
French (native), English (upper-intermediate), Dutch (lower-intermediate).


\end{document}
